%% Problemstellung.tex
%% $Id: Problemstellung.tex 61 2012-05-03 13:58:03Z bless $
%%


\chapter{Problemstellung und eigenes Erkenntnisinteresse}
\label{ch:Problemstellung}

Aufgrund von aktuellen Gegebenheiten bei der Deutschen Telekom AG mit Kostensenkungen und Umstrukturierung oder dem Stellenabbau bei T –Systems \cite{web:Handelsblatt}, stellt sich die Frage nach der Anwendung des Controllings sowie den Unterschied zwischen Theorie, Konzeption und Praxis.
\\
\\
Durch den Einsatz des Autors als dualer Student bei T-Systems im Service Delivery Management wurde dessen Arbeitsumfeld erlebt.
Die Aufgabe im Service Delivery Management ist nach ITIL unter anderem die Begleitung des IT Service Lebenszyklus \cite{web:LiveCylce} in all deren Phasen.
Von Service Strategie, über Service Design, Service Transition, Service Operation bis hin zur Kontinuierlichen Service Verbesserung.
\\
\\
Hier tritt das erste Problem auf, denn das Ziel vieler Unternehmen ist es Kunden langfristig an sich zu binden um nachhaltige Wettbewerbsvorteile zu schaffen. 
\\
Um dem internationalen Druck der Konkurrenz im IT Sourcing standzuhalten, ist neben einem Umfangreichen Portfolio, durchgängiger Verfügbarkeit und umfassenden "Know How", eine Finanzielle Strategie für den gesamten Lebenszyklus zu entwickeln.
Dabei ist es fundamental die erarbeitete Strategie im Anschluss zu steuern.
Es gilt ein Zielsystem für die Budgetierung zu planen und dabei Planungsschritte und Unterlagen zu erstellen.
Des Weiteren ein Informationssystem für das Management zu organisieren, um das Fundament zur Beurteilung der Wirtschaftlichkeit und Geschäftsentwicklung an Kennzahlen der Betriebswirtschaftslehre zu legen.
Im Nachhinein wird die Entwicklung an aufgestellten Kennzahlen überprüft und mit festgelegten Maßnahmen gesteuert.
Hierbei werden unter anderem Soll-Ist-Vergleiche durchgeführt und die Ursachen für Abweichungen vom Controller ermittelt und analysiert um Prognosen abgeben zu können.
Die definierten Maßnahmen werden vom Controller koordiniert und in den einzelnen Teilbereichen angepasst.
\\
\\
Um mit internen sowie externen Kosten für die Erbringung des Services kalkulieren einzuhalten und die Notwendigkeit positive Geschäftszahlen erreichen zu können, werden die dargestellten Controlling \cite{web:Controlling} Aufgaben ebenfalls innerhalb des Service Delivery Managements übernommen.
\\
\\
Die dargestellten Informationen und Probleme veranschaulichen Bedarf das Kostencontrolling von T-Systems näher zu untersuchen.




%%% Local Variables: 
%%% mode: latex
%%% TeX-master: "thesis"
%%% End: 
