%% Fragestellung.tex
%% $Id: Fragestellung.tex 61 2012-05-03 13:58:03Z bless $
%%

\chapter{Fragestellung oder Hypothese}
\label{ch:Fragestellung}

Dieses Kapitel beschäftigt sich mit den einzelnen Fragestellungen die sich bis hin zur Bachelorthesis hin entwickelt. 
Aus den in Punkt ein und Punkt zwei genannten Theoretischen Grundlagen sowie den unterschiedlichen Einsatzbereichen des Controllings ergeben sich folgende Fragstellungen:
\begin{quote}
1. Fragestellung: 
Welche Philosophie beinhaltet das Controlling?
\end{quote}

Dieser Abschnitt soll die Problemstellung verdeutlichen und die Controlling Theorie, Controlling Konzeptionen und Philosophie des Kostencontrolling sowie der Hintergrund verdeutlichen.
Aufgrund der Beantwortung dieser Frage und die somit verbundene Verdeutlichung des Themas, ergibt sich folgende Fortsetzung:
\begin{quote}
2. Fragestellung: 
Welche Treiber gibt es in der deutschen Wirtschaft für Kostencontrolling?
\end{quote}

Hier werden die treibenden Unternehmen und Anreize für Kostencontrolling der deutschen Wirtschaft wiedergespiegelt. Darüber hinaus sollen mögliche Tools und Software für Controlling erläutert werden sowie die Controlling Instrumente und Anwendungsbereiche in Unternehmen aufgezeigt werden.
Nach diesen Erläuterungen soll auf den Spezialfall Bezug genommen werden, wodurch so folgende Fragestellung ergibt
\begin{quote}
3. Fragestellung: 
Wie ist das Controlling bei T-Systems aufgebaut?
\end{quote}

Diese Untersuchung gibt Aufschluss über das Profil der T-Systems und deren Anforderungen an das Kostencontrolling. Des Weiteren soll dieser Prozess wiedergespiegelt sowie dessen Umsetzung mit Hilfe von Angewandten Tools beschrieben werden.
\begin{quote}
4. Fragestellung: 
Wie gestaltet sich die Anwendung am Kundenbeispiel?
\end{quote}

Anschließend wird die theoretische Auslegung des Kostencontrollings bei T-Systems anhand eines gezielten Kundenbeispiels beschrieben. Dies wird anhand einer empirischen Erhebung des IST-Zustandes mithilfe des Kundenbeispiels beschrieben. Dadurch ergibt sich die grundlegende Fragestellung der Arbeit:
\begin{quote}
5. Fragestellung: 
Welche Unterschiede ergeben sich zwischen Theorie und Praxis innerhalb der T-Systems?
\end{quote}

Hier werden die Unterschiede und Abweichung zwischen Theorie, Konzeption und Praxis erläutert und abschließend in Risiken und Benefiz sowie einem Ausblick auf weitere Fragestellungen beendet.


%%% Local Variables: 
%%% mode: latex
%%% TeX-master: "thesis"
%%% End: 
