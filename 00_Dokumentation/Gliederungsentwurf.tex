%% Gliederungsentwurf.tex
%% $Id: Gliederungsentwurf.tex 61 2012-05-03 13:58:03Z bless $

\chapter{Gliederungsentwurf}
\label{ch:Gliederungsentwurf}
%% ==============================

Dieses Kapitel enthält die vorläufige Gliederung der Bachelorthesis und Zeigt den aktuellen IST-Zustand. Im Verlauf bis hin zur endgültigen Abgaben kann sich dieser Entwurf noch geringfügig andern. Der Gliederungsentwurf soll die aufgeführten Leitfaden in Punkt 3 beinhalten und diese beantworten.

\begin{center}
\begin{description}
\item[Inhaltsverzeichnis]~\par
\begin{enumerate}
	\item Einleitung
	\begin{enumerate}
         \item Problemstellung 
         \item Zielsetzung
         \item Aufbau und Vorgehensweise der Arbeit
    \end{enumerate}     
	\item Grundlagen zum Kostencontrolling
	\begin{enumerate}
         \item Controlling Theorie 
         \item Controlling Instrumente
         \item Controlling Konzeption
         \item Philosophie / Hintergrund Kostencontrolling
    \end{enumerate}     
	\item Antreiber innerhalb der deutschen Wirtschaft
	\begin{enumerate}
         \item Unternehmen
         \item Tools
         \item Anwendungsbereiche
    \end{enumerate}
	\item Hintergrund T-Systems
	\begin{enumerate}
         \item Unternehmensprofil
         \item Anforderungen an Kostencontrolling
         \item Prozesssichtweise der Umsetzung
         \item Angewandte Tools
    \end{enumerate}
	\item Kundenbeispiel
	\begin{enumerate}
         \item Anwendung
         \item Vorgehen
         \item Kennzahlen
    \end{enumerate}
	\item Vergleiche
	\begin{enumerate}
         \item Praxis – Theorie
         \item Praxis – Konzeption 
    \end{enumerate}
	\item Zusammenfassung
	\begin{enumerate}
         \item Risiken \& Benefiz
         \item Fazit
    \end{enumerate}
\end{enumerate} 
\end{description}	
\end{center}
  
%%% Local Variables: 
%%% mode: latex
%%% TeX-master: "thesis"
%%% End: 
