%% Methodik.tex
%% $Id: Methodik.tex 61 2012-05-03 13:58:03Z bless $
%%

\chapter{Methodik und Material}
\label{ch:Methodik}

Nachfolgend wird die Vorgehensweise zur Bearbeitung der Bachelorarbeit aufgeführt: 
\begin{center}
\begin{description}
\item[Inhaltsverzeichnis]~\par
\begin{itemize}
	\item Erklärung und Darstellung des Begriffes Controlling
	\begin{itemize}
		\item Darstellung der Theorie
		\item Darstellung der Instrumente
		\item Darstellung der Konzeptionen
		\item Erklärung der Philosophie von Kostencontrolling
	\end{itemize}
	\item Identifizierung von Antreibern in der deutschen Wirtschaft
	\begin{itemize}
		\item Identifizierung und Unternehmen
		\item Identifizierung und Beschreibung von möglichen Tools
		\item Identifizierung und Darstellung von Anwendungsbereichen
	\end{itemize}
	\item Erklärung und Darstellung der T-Systems
	\begin{itemize}
		\item Beschreibung des Unternehmensprofils
		\item Beschreibung der Anforderungen an das Kostencontrolling
		\item Beschreibung der Prozesssichtweise und geplanten Umsetzung
		\item Beschreibung der verwendeten Tools
	\end{itemize}
	\item Empirische Untersuchung
	\begin{itemize}
		\item Dokumentation und Beschreibung des angewandten Kundenprozesses
		\item Beschreibung der Umsetzung des Spezialfalls
		\item Beschreibung der Kennzahlen
	\end{itemize}
	\item Vergleiche
	\begin{itemize}
		\item Identifizierung und Beschreibung von Unterschieden zwischen Theorie und Praxis
		\item Identifizierung und Beschreibung von Unterschieden zwischen Theorie und Konzeption
	\end{itemize}
	\item Ermittlung der Zustimmung oder Ablehnung der formulierten Hypothesen
\end{itemize} 
\end{description}	
\end{center}

Im Mittelpunkt der Arbeit stehen die Erfassung der Theorie des Kostencontrollings sowie die empirische Analyse des Angewandten Prozesses bei T-Systems am Kundenbeispiel. Darüber hinaus liegt der Fokus auf dem Vergleich dieser ermittelten Informationen. 


%%% Local Variables: 
%%% mode: latex
%%% TeX-master: "thesis"
%%% End: 
